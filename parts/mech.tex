\chapter{Mechanism of Action}
Benzodiazepines induce a calming effect by increasing the efficiency of a natural brain chemical\emph{GABA} which leads to a decrease in the excitability of neurons.

GABA$_A$ receptors are heteropentameric membrane proteins that form a chloride channel. They are composed of several subunits ($\alpha 1-6, \beta 1-3, \gamma 1-3, \delta, \epsilon, \theta, \pi$, and $\rho 1-4$). The \emph{GABA} receptors are sensitive to benzodiazepines specifically those that contain the $\alpha(1,2,3,5)$, $\beta(2,3)$ and $\gamma 2$ subunits in a $2:2:1$ stoichiometry.

\emph{GABA} binds to the \emph{GABA$_A$} receptors and open the chloride channels and causes a change in potential of the membrane. Benzodiazepines interact allosterically with the \emph{GABA$_A$} and increase the affinity of \emph{GABA} to the \emph{GABA$_A$} receptors. The change at the molecular level is that when benzodiazepines bind to the \emph{GABA$_A$} receptors they prolong the duration at which the \emph{GABA} binds to the its receptor and thus increase the amount of charge that being transferred, this leads to an increase in potential across the membrane of the cell.


The \emph{GABA} itself which is the main drug that induces the action activate the \emph{GABA$_A$} receptor which upon activation conducts chloride ions selectively across the membrane which results in hyperpolarization thus inhibiting the neurotransmission by reducing the potential of having an action-potential. 

Benzodiazepine stimulate the sedative effect by acting on the \emph{GABA$_A-\alpha 1$} receptor\cite{rudolph1999benzodiazepine} and stimulate anxiolytic\footnote{anti-anxiety} by interacting with the \emph{GABA$_A-\alpha 2$} receptor.\cite{kopp2004modulation} 




