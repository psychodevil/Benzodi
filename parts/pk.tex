\chapter{Pharmacokinetics}

\section{Chlordiazepoxide}
%Draw chemical structure of the metabolites in the metabolites section
The complexity of chlordiazepoxide arise from the pharmacologically actie metabolites that it gets transformed into such as desmethylchlordiazepoxide, demoxepam, desmethyldiazepam, and oxazepam.\footnote{The chemical structures are to be found in the metabolites chapter \ref{sec:met:chlor}}

The elimination half-life (t$\frac{1}{2}\beta$) of single doses of chlordiazepoxide in healthy individuals ranges from from $5$ to $30$ hours, the volume of distrubtion ($V_d$) ranges from $0.25$ to $0.50$ liters/kg. The hepatic extraction ratio is under $5\%$.

Clearance of chlordiazepoxide is reduced in the elderly, patients with cirrhosis\footnote{an abnormal liver condition in which there is irreversible scarring of the liver.}, and patients receiving concurrent medicationof disulfiram\footnote{known as Antabus and is used in treatment of alcoholism}. Oral doses are absorbed completely and rapidly, while intramuscular injection is painful and results in slow unpredictable absorption.\cite{Greenblatt1978}  

\section{Diazepam}
\subsection{Absorption}
Oral bioavailibity of diazepam is about $0.9$ and the average time to achieve a peak in plasma concentration (T$_{max}$) ranges from 1 to $1.5$ hours. High fat food increases increases absorption lag time to about $45$ minutes as compared with $15$ minutes for a fasting person, and an increase in T$_{max}$ to about $2.5$ hours. C$_{max}$ also drops by $20\%$ and AUC drops by about $27\%$ (range $15\%$ to $50\%$).

\subsection{Distribution}
Diazepam along with its metabolites show high affinity to plasma proteins ($98\%$). Diazepam along with its metabolites have a high lipophilicity and thus cross the blood-brain barrier along with the placenta, and is found in breast milk in concentrations of about $\frac{1}{10}$ of the maternal plasma.
The volume of distribution (V$_d$) at steady state is about $0.8$ to $1.0$ L/kg. Diazepam shows a biphasic plasma concentration-time profile after oral administration. The  distribution phase has a half-life of about $1$ to $>3$hours.

\subsection{Metabolism}
Diazepam tranforms to nordiazepam (gets N-methylated) by [Cytochrome P450 2C9, Cytochrome P450 2B6, Cytochrome P450 2C19, Cytochrome P450 3A5, Cytochrome P450 3A4],\cite{zhou2009substrates} gets metabolized to temazepam by [Prostaglandin G/H synthase 1, Cytochrome P450 2C19], and to oxazepam by cytochrome P$450$ $1$A$2$, it also gets metabolized to desmehtyldiazepam.

Noridazepam gets metabolized to oxazepam by [Cytochrome P450 2C19, Cytochrome P450 3A4, Cytochrome P450 3A5], it also gets metabolized to Nordiazepam O-glucuronide.

Tamazepam gets metabolized to Oxazepam by [Cytochrome P450 2B6, Cytochrome P450 2C9, Cytochrome P450 2C19, Cytochrome P450 3A4, Cytochrome P450 3A5]

\subsection{Elimination}
The elimination phase following the distribution phase has a half-life of about $48$ hours. The eliminatino half-life time of the active metabolite N-desmethyldiazepam is up to $100$ hours. Diazepams along with its metabolites are excreted in the urine as their glucuronide conjugates. The clearance of diazepam is $20$ to $30$ mL/min. 

\section{Lorazepam}
Lorazepam as a 3-hydroxy benzodiazepine derivative has a major metabolic pathway that involes conjugation to glucuronic acid at position 3. The glucuronide metabolite is inactie and is followed by urinary excretion.

Elimination half-life time ranges between 5 and 15 hours in healthy individuals. The volume of distribution (V$_d$) ranges between 0.6 to 2.0 L/kg which indicates a very high affinity to plasma proteins. Clearance is from $0.9$ to $2.0$ ml/min/kg. Aging and liver diseases do not have a substantial influence on the kinetics of lorazepam, while relan diseases are associated with prolonged half-life and increased volume of distribution.\cite{Greenblatt1981}
